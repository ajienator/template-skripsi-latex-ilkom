%-------------------------------------------------------------------------------
%                      Template Naskah Skripsi Ilmu Komputer
%               	Berdasarkan format Teknik Informatika FMIPA UNNES
% 						(c) Aji Purwinarko 
%-------------------------------------------------------------------------------

%Template pembuatan skripsi.
\documentclass{skripsi}

% prefiks pada daftar gambar dan tabel
\usepackage[titles]{tocloft}
\renewcommand\cftfigpresnum{Gambar\  }
\renewcommand\cfttabpresnum{Tabel\   }

% hyperlink dan table of content
\usepackage[hidelinks]{hyperref}

\newlength{\mylenf}
\settowidth{\mylenf}{\cftfigpresnum}
\setlength{\cftfignumwidth}{\dimexpr\mylenf+2em}
\setlength{\cfttabnumwidth}{\dimexpr\mylenf+2em}

%Untuk Bold Face pada Keterangan Gambar
\usepackage[labelfont=bf]{caption}

%Untuk caption dan subcaption
\usepackage{caption}
\usepackage{subcaption}


\usepackage[utf8]{inputenc}
\usepackage[english]{babel}
\usepackage{enumitem}
\usepackage{apacite}
\usepackage{natbib}
\bibliographystyle{UNNESapa}
\AtBeginDocument{
	\renewcommand{\BBOP}{}
	\renewcommand{\BBCP}{}
}
\setcitestyle{authoryear, open={(},close={)},notesep={: }}

%\usepackage[none]{hyphenat}
% setting hyphenation pada paragraf
\tolerance=1
\emergencystretch=\maxdimen
\hyphenpenalty=10000
\hbadness=10000

%\newcommand\listappendixname{DAFTAR LAMPIRAN}
%\newcommand\appcaption[1]{%
%   \addcontentsline{app}{chapter}{#1}}
%\makeatletter
%\newcommand\listofappendices{%
%   \chapter*{\listappendixname}\@starttoc{app}}
%\makeatother

\newcommand{\listappendicesname}{DAFTAR LAMPIRAN}
\newlistof{appendices}{apc}{\listappendicesname}
\newcommand{\appendices}[1]{\addcontentsline{apc}{appendices}{#1}}
\newcommand{\newappendix}[1]{\section*{#1}\appendices{#1}}
%\makeatletter
%\renewcommand\section{\@startsection {section}{1}{\z@}%
%  {-3.5ex \@plus -1ex \@minus -.2ex}%
%  {2.3ex \@plus.2ex}%
%  {\centering\normalfont\Large\bfseries}
%  }
%\makeatother

%-----------------------------------------------------------------
%Disini awal masukan untuk data proposal skripsi
%-----------------------------------------------------------------

\degree{Sarjana Komputer}
\yearsubmit{2019}
\program{Teknik Informatika}
\dept{Ilmu Komputer}


\titleind{Penerapan \textit{Discretization} dan \textit{Correlation Based Feature Selection} untuk \textit{Optimasi Support Vector Machine} dalam \textit{Diagnosis Chronic Kidney Disease}}

\fullname{Pipit Riski Setyorini}
\idnum{4611413041}


% Tanggal Pernyataan
\statementdate{Agustus 2019}

% Tanggal Persetujuan sidang
\agreementdate{Agustus 2019}

% Tanggal sidang skripsi
\approvaldate{3 Agustus 2019}

% Panitia Ujian
% Ketua
\dean{Dr Sugianto M.Si}
\deannip{1961 0219 1993 03 1 001}
% Sekretaris
\secretary{Endang Sugiharti S.Si.,M.Kom}
\secretarynip{1974 0107 1999 03 2 001}
% Penguji 1
\examinera{Riza Arifudin S.Pd., M.Cs.}
\examinernipa{1980 0525 2005 01 1 001}
% Penguji 2
\examinerb{Endang Sugiharti S.Si.,M.Kom}
\examinernipb{1974 0107 1999 03 2 001}

% Pembimbing Utama
\firstsupervisor{Much Aziz Muslim S.Kom., M.Kom.}
\firstnip{1974 0420 2008 12 1 001}

\secondsupervisor{}
\secondnip{}

%-----------------------------------------------------------------
%Disini akhir masukan untuk data proposal skripsi
%-----------------------------------------------------------------

\begin{document}

\cover

\statementpage

\agreementpage

\approvalpage

%-----------------------------------------------------------------
%Disini awal masukan MOTTO dan PERSEMBAHAN
%-----------------------------------------------------------------
%!TEX root = ./template-skripsi.tex
%-------------------------------------------------------------------------------
%               		MOTTO dan PERSEMBAHAN
%-------------------------------------------------------------------------------
\acknowledgment
 \noindent  \MakeUppercase{\normalfont\bfseries MOTTO}
\noindent  \begin{itemize}[leftmargin=*,noitemsep,topsep=0pt]
	\item Every pain has a reason.
	\item Jangan pernah kita meninggalkan do’a untuk ibu bapak karena itu adalah kunci pembuka pintu rezeki kita
	\item Karena sesungguhnya sesudah kesulitan itu ada kemudahan. (Q.S. Al-Insyirah:5)
	\item Kehilangan mengajarkanmu keikhlasan.
\end{itemize}

\vspace{1.5cm}
\noindent  \begin{tabular}{p{2.5cm} p{10.5cm}}
& \MakeUppercase{\normalfont\bfseries PERSEMBAHAN} \\
& Skripsi ini ku persembahkan kepada: \\
& \vspace*{-\baselineskip} \noindent  \begin{itemize}[leftmargin=*, noitemsep,topsep=0pt]   
	\item Kedua orang tuaku yang sangat saya cintai, Ibu Karni dan Bapak Sudono, terimakasih atas do’a, dukungan dan kasih sayang yang tiada hentinya selalu engkau berikan.
	\item Kakak-kakakku tercinta, Eka Kurniawati dan Slamet Raharjo yang selalu memberiku semangat dan doa.
	\item Sahabat-sahabat ILKOM, SMAN 1 Pati dan teman-teman kos yang telah memberiku semangat dan saran.
	\item Semua pihak yang tidak dapat disebutkan satu persatu yang telah membantu hingga terselesaikannya penulisan skripsi ini.
	\item Almamaterku UNNES 
	\end{itemize}	\\
\end{tabular}

%\begin{flushright}
%\emph{Untuk Ibu, Bapak,\\dan Adik-adikku tercinta.}
%\end{flushright}

%-----------------------------------------------------------------
%Disini awal masukan untuk Prakata
%-----------------------------------------------------------------
%!TEX root = ./template-skripsi.tex
%-------------------------------------------------------------------------------
%               		PRAKATA
%-------------------------------------------------------------------------------
\preface
Assalamu'alaikum Wr. Wb.

\vspace{0.5cm}

Puji syukur penulis panjatkan ke hadirat Allah SWT karena hanya dengan rahmat dan hidayah-Nya, Tugas Akhir ini dapat terselesaikan tanpa halangan berarti. Keberhasilan dalam menyusun laporan Tugas Akhir ini tidak lepas dari bantuan berbagai pihak yang mana dengan tulus dan ikhlas memberikan masukan guna sempurnanya Tugas Akhir ini. Oleh karena itu dalam kesempatan ini, dengan kerendahan hati penulis mengucapkan terima kasih kepada:

\begin{enumerate}
\item{Bapak Sarjiya, S.T., M.T., Ph.D., selaku Ketua Jurusan Teknik Elektro dan Teknologi Informasi Fakultas Teknik Universitas Gadjah Mada,}
\item{Bapak Sigit Basuki Wibowo, S.T., M.Eng. selaku dosen pembimbing pertama yang telah memberikan banyak bantuan, bimbingan, serta arahan dalam Tugas Akhir ini,}
\item{Bapak Bimo Sunarfri Hantono, S.T., M.Eng. selaku dosen pembimbing kedua yang juga telah memberikan banyak bantuan, bimbingan, serta arahan dalam Tugas Akhir dan kegiatan-kegiatan yang lain,}
\item{Bapak Warsun Najib, S.T., M.Sc. selaku dosen pembimbing akademis penulis dan juga dosen pembimbing lapangan penulis pada KKN-PPM UGM 2013 Unit SLM07,}
\item{Seluruh Dosen di Jurusan Teknik Elektro dan Teknologi Informasi FT UGM, yang tidak bisa disebutkan satu-satu, atas ilmu dan bimbingannya selama penulis berkuliah di JTETI,}
\item{Ibu dan Bapak yang selama ini telah sabar membimbing, mengarahkan, dan mendoakan penulis tanpa kenal lelah untuk selama-lamanya, dan}
\item{Cantumkan pihak-pihak lain yang ingin anda berikan ucapan terimakasih.}
\end{enumerate}

Penulis menyadari bahwa penyusunan Tugas Akhir ini jauh dari sempurna. Kritik dan saran dapat ditujukan langsung pada e-mail atau \emph{mention} langsung pada akun \emph{twitter} saya. Akhir kata penulis mohon maaf yang sebesar-besarnya apabila ada kekeliruan di dalam penulisan Tugas Akhir ini.

\vspace{0.5cm}

Wassalamu'alaikum Wr. Wb.

\begin{tabular}{p{7.5cm} l}
& Semarang, Agustus 2019 \\
&\\
&\\
&Pipit Riski Setyorini \\
&4611413041
\end{tabular}

%-----------------------------------------------------------------
%Disini awal masukan Intisari
%-----------------------------------------------------------------
%!TEX root = ./template-skripsi.tex
%-------------------------------------------------------------------------------
%               		ABSTRAK
%-------------------------------------------------------------------------------
\begin{abstractind}

\noindent
Pipit Riski Setyorini. 2019. Penerapan Discretization dan Correlation Based Feature Selection untuk Optimasi Support Vector Machine dalam Diagnosis Chronic Kidney Disease. Skripsi, Jurusan Ilmu Komputer Fakultas Matematika dan Ilmu Pengetahuan Alam Universitas Negeri Semarang. Pembimbing Utama Much Aziz Muslim, S.Kom., M.Kom. dan Pembimbing Pendamping Endang Sugiharti, S.Si., M.Kom.\\
\vspace{0.5cm}


\noindent
\textbf{Kata kunci :} \emph{wireless sensor network}, \emph{Internet Protocol}, WiFi, interoperabilitas. \\

\vspace{0.5cm}
\noindent
Lorem ipsum dolor sit amet, consectetur adipisicing elit, sed do eiusmod tempor incididunt ut labore et dolore magna aliqua. Ut enim ad minim veniam, quis nostrud exercitation ullamco laboris nisi ut aliquip ex ea commodo consequat. Duis aute irure dolor in reprehenderit in voluptate velit esse cillum dolore eu fugiat nulla pariatur. Excepteur sint occaecat cupidatat non proident, sunt in culpa qui officia deserunt mollit anim id est laborum.

\noindent
Sed ut perspiciatis unde omnis iste natus error sit voluptatem accusantium doloremque laudantium, totam rem aperiam, eaque ipsa quae ab illo inventore veritatis et quasi architecto beatae vitae dicta sunt explicabo. Nemo enim ipsam voluptatem quia voluptas sit aspernatur aut odit aut fugit, sed quia consequuntur magni dolores eos qui ratione voluptatem sequi nesciunt.

\end{abstractind}


%-----------------------------------------------------------------
%Disini akhir masukan untuk muka skripsi
%-----------------------------------------------------------------
\tableofcontents
\addcontentsline{toc}{chapter}{DAFTAR ISI}
\listoftables
\addcontentsline{toc}{chapter}{DAFTAR TABEL}
\listoffigures
\addcontentsline{toc}{chapter}{DAFTAR GAMBAR}

\addcontentsline{toc}{chapter}{DAFTAR LAMPIRAN}
%daftar lampiran
\listofappendices
\selectlanguage{bahasa}\clearpage\pagenumbering{arabic}\setcounter{page}{1}


\include{bab1}

\include{bab2}

\include{bab3}

\include{bab4}

\include{bab5}

%-----------------------------------------------------------------
%Disini akhir masukan Bab
%-----------------------------------------------------------------


%-----------------------------------------------------------------
% Disini awal masukan untuk Daftar Pustaka
% - Daftar pustaka diambil dari file .bib yang ada pada folder ini
%   juga.
% - Untuk memudahkan dalam memanajemen dan menggenerate file .bib
%   gunakan reference manager seperti Mendeley, Zotero, EndNote,
%   dll.
%-----------------------------------------------------------------

\bibliography{collection}
\addcontentsline{toc}{chapter}{DAFTAR PUSTAKA}

%-----------------------------------------------------------------
%Disini akhir masukan Daftar Pustaka
%-----------------------------------------------------------------

%!TEX root = ./template-skripsi.tex
%-------------------------------------------------------------------------------
%               		LAMPIRAN
%-------------------------------------------------------------------------------

\addcontentsline{toc}{chapter}{LAMPIRAN}
  \chapter*{LAMPIRAN}%

\appendix
\newappendix{Lampiran 1}
Lorem ipsum is a pseudo-Latin text used in web design, typography, layout, and printing in place of English to emphasise design elements over content. 
		
\newappendix{Lampiran 3} 
 It's also called placeholder (or filler) text. It's a convenient tool for mock-ups. 
		
\newappendix{Lampiran 3} 
 It helps to outline the visual elements of a document or presentation, eg typography, font, or layout. Lorem ipsum is mostly a part of a Latin text by the classical author and philospher Cicero.

\newappendix{Lampiran 4}
Its words and letters have been changed by addition or removal, so to deliberately render its content nonsensical; it's not genuine, correct, or comprehensible Latin anymore



	\begin{enumerate}
		\item Lorem ipsum is a pseudo-Latin text used in web design, typography, layout, and printing in place of English to emphasise design elements over content. 
		
		\item It's also called placeholder (or filler) text. It's a convenient tool for mock-ups. 
		
		\item It helps to outline the visual elements of a document or presentation, eg typography, font, or layout. Lorem ipsum is mostly a part of a Latin text by the classical author and philospher Cicero.

		\item Its words and letters have been changed by addition or removal, so to deliberately render its content nonsensical; it's not genuine, correct, or comprehensible Latin anymore. 
	\end{enumerate}

	



\end{document}

