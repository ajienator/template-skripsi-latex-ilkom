%!TEX root = ./template-skripsi.tex
%-------------------------------------------------------------------------------
%               		MOTTO dan PERSEMBAHAN
%-------------------------------------------------------------------------------
\acknowledgment
 \noindent  \MakeUppercase{\normalfont\bfseries MOTTO}
\noindent  \begin{itemize}[leftmargin=*,noitemsep,topsep=0pt]
	\item Every pain has a reason.
	\item Jangan pernah kita meninggalkan do’a untuk ibu bapak karena itu adalah kunci pembuka pintu rezeki kita
	\item Karena sesungguhnya sesudah kesulitan itu ada kemudahan. (Q.S. Al-Insyirah:5)
	\item Kehilangan mengajarkanmu keikhlasan.
\end{itemize}

\vspace{1.5cm}
\noindent  \begin{tabular}{p{2.5cm} p{10.5cm}}
& \MakeUppercase{\normalfont\bfseries PERSEMBAHAN} \\
& Skripsi ini ku persembahkan kepada: \\
& \vspace*{-\baselineskip} \noindent  \begin{itemize}[leftmargin=*, noitemsep,topsep=0pt]   
	\item Kedua orang tuaku yang sangat saya cintai, Ibu Karni dan Bapak Sudono, terimakasih atas do’a, dukungan dan kasih sayang yang tiada hentinya selalu engkau berikan.
	\item Kakak-kakakku tercinta, Eka Kurniawati dan Slamet Raharjo yang selalu memberiku semangat dan doa.
	\item Sahabat-sahabat ILKOM, SMAN 1 Pati dan teman-teman kos yang telah memberiku semangat dan saran.
	\item Semua pihak yang tidak dapat disebutkan satu persatu yang telah membantu hingga terselesaikannya penulisan skripsi ini.
	\item Almamaterku UNNES 
	\end{itemize}	\\
\end{tabular}